%%%%%%%%%%%%%%%%%%%%%%%%%%%%%%%%%%%%%%%%%%%%%%%%%%%%%%%%%%%%%%%%%%%%%%%%%%%%%%%%

\begin{abstract}\thispagestyle{empty}
    En este documento se relata el proceso de creación que hay detrás de un controlador mio-eléctrico para una prótesis de miembro superior con amputación a nivel de mano.
    Dicho controlador leerá las señales \textit{EMG} que se producen en el paciente mediante electrodos colocados en el flexor carpi radialis y en el extensor carpi radialis longus,  músculos implicados en la pronación-supinación del antebrazo y flexo-extensión de la muñeca. Las señales serán filtradas por un controlador \textit{ARDUINO UNO} para finalmente ser procesadas por una red de neuronas primeramente simulada en Matlab y posteriormente implementada en \textit{FPGA}.
    % substitúe este comando polo resumo do teu TFG
    % na lingua principal do documento (tipicamente: galego)

  \vspace*{25pt}
  \begin{segundoresumo}
    This document describes the creation process behind a myoelectric controller for an upper limb prosthesis with hand-level amputation.
    The controller will read the \textit{EMG} signals that are produced in the patient by  electrodes placed on the flexor carpi radialis and on the extensor carpi radialis longus, muscles involved in pronation-supination of the forearm and flexion-extension of the wrist. 
    Signals are filtered by a microcontroller \textit{ARDUINO UNO} to finally be processed on a network of neurons first simulated in \textit{Matlab} and later implemented in \textit{FPGA}. 
    % substitúe este comando polo resumo do teu TFG
    % na lingua secundaria do documento (tipicamente: inglés)
  \end{segundoresumo}
\vspace*{25pt}
\begin{multicols}{2}
\begin{description}
\item [\palabraschaveprincipal:] \mbox{} \\[-20pt]
  \blindlist{itemize}[7] % substitúe este comando por un itemize
                         % que relacione as palabras chave
                         % que mellor identifiquen o teu TFG
                         % no idioma principal da memoria (tipicamente: galego)
\end{description}
\begin{description}
\item [\palabraschavesecundaria:] \mbox{} \\[-20pt]
  \blindlist{itemize}[7] % substitúe este comando por un itemize
                         % que relacione as palabras chave
                         % que mellor identifiquen o teu TFG
                         % no idioma secundario da memoria (tipicamente: inglés)
\end{description}
\end{multicols}

\end{abstract}

%%%%%%%%%%%%%%%%%%%%%%%%%%%%%%%%%%%%%%%%%%%%%%%%%%%%%%%%%%%%%%%%%%%%%%%%%%%%%%%%
