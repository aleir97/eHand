\chapter{Contido demostrativo}
\label{chap:demo}

\lettrine{E}{ntre} a introdución e as conclusións, o documento conterá
tantos capítulos como sexa preciso, sempre con coidado de non rebasar
o límite de 80 páxinas fixado polo regulamento de TFGs.

Empregaremos éste de xeito demostrativo, para ilustrar o uso de
elementos habituais que poidan ser de utilidade\footnote{Por exemplo,
  isto é unha nota a pé de páxina.}.

\section{Inclusión de imaxes}

Se precisamos imaxes no noso documento, incluirémolas do xeito que se
indica na figura~\ref{fig:exemplo} (páxina~\pageref{fig:exemplo}). Se
o facemos así, \LaTeX ubicará cada imaxe no mellor lugar posible,
lugar que pode variar a medida que o documento vaia crecendo coa
inclusión de máis texto e outros elementos (máis imaxes, táboas,
etc.).

\begin{figure}[hp!]
  \centering
  \includegraphics[width=0.75\textwidth]{imaxes/udc.png}
  \caption{Pé de imaxe descritivo}
  \label{fig:exemplo}
\end{figure}

Recoméndase almacenar os ficheiros gráficos no directorio
\texttt{imaxes}.

\subsection{Inclusión de varias sub-imaxes}

Se precisamos inserir imaxes relacionadas, pode ser apropiado
incluílas como sub-figuras, do xeito que se pode apreciar na
figura~\ref{fig:exemplo-subfiguras} coas
imaxes~\ref{fig:subfigura-rotada}
e~\ref{fig:subfigura-deformada}. Como se pode ver nos exemplos desta
sección, sempre é recomendable referirse ás imaxes pola súa
referencia, xa que dese xeito non dependemos de onde queden ubicados
os elementos en cuestión.

\begin{figure}[hp!]
  \centering
  \begin{subfigure}[c]{0.3\textwidth}
    \includegraphics[angle=45,width=\textwidth]{imaxes/udc.png}
    \caption{Pé de subimaxe rotada}
    \label{fig:subfigura-rotada}
  \end{subfigure}
  \hspace{0.1\textwidth}
  \begin{subfigure}[c]{0.3\textwidth}
    \includegraphics[width=\textwidth,height=3cm]{imaxes/udc.png}
    \caption{Pé de subimaxe deformada}
    \label{fig:subfigura-deformada}
  \end{subfigure}
  \caption{Pé de imaxe xeral}
  \label{fig:exemplo-subfiguras}
\end{figure}

\section{Inclusión de código fonte}

Se precisamos incluír fragmentos de código fonte, podemos facelo da
seguinte maneira:

\begin{lstlisting}[language=C]
#include <stdio.h>
#define N 10

int main()
{
  int i;

  // Isto é un comentario
  puts("Ola, mundo!");

  for (i = 0; i < N; i++)
  {
    puts("LaTeX é a ferramenta de edición ideal para profesionais da informática!");
  }

  return 0;
}
\end{lstlisting}

\section{Uso da relación de acrónimos e do glosario}

Os acrónimos edítanse no ficheiro \texttt{bibliografia/acronimos.tex}
e úsanse empregando a orde \texttt{acrlong} para obter o termo
completo (deste xeito: \acrlong{erlang}), a orde \texttt{acrshort}
para obter o acrónimo (deste xeito: \acrshort{erlang}). A primeira vez
que usamos un termo con acrónimo no documento é recomendable usar orde
\texttt{acrfull} (que produce ambas versións á vez:
\acrfull{erlang}). Os acrónimos que non se usan no documento, non
aparecen na relación que se xerar na versión PDF.

Pola súa banda, os termos do glosario edítanse no ficheiro
\texttt{bibliografia/glo\-sa\-rio.tex} e úsanse empregando a orde
\texttt{gls} (deste xeito, \gls{bytecode}) ou \texttt{Gls} (deste
xeito, \Gls{bytecode}). Ao igual que os acrónimos, os termos que non
se usan no documento, non aparecen na relación que se xera na versión
PDF.
